\documentclass[11pt]{article}

\usepackage{helvet}
\renewcommand{\familydefault}{\sfdefault}

\usepackage{amsmath,amssymb,amsthm}
\usepackage{graphicx}
\usepackage{subfigure}
\usepackage{caption}

\usepackage[top=1in,bottom=1in,left=1in,right=1in]{geometry}
\usepackage{paralist}

\usepackage[pagebackref=true,colorlinks=true,breaklinks=true,linkcolor=black,citecolor=black,urlcolor=black]{hyperref}

\usepackage{fancyhdr}
\setlength{\headheight}{15.2pt}
\pagestyle{fancyplain}
\lhead[]{} 
\chead[hTeXML: Web Authoring for STEM]{SHORT\_RUNNING\_TITLE}
\rhead[C. Aguilar]{C. Aguilar}

% Move sections heading to the center
\usepackage[center]{titlesec}
\titleformat{\section}[hang]{\normalfont\scshape}{\thesection.}{.5em}{\filcenter}[]
\titleformat{\subsection}[hang]{\normalfont\scshape}{\thesubsection.}{.5em}{\filcenter}[]

% For tables
\usepackage{array}
\usepackage{color, colortbl}
\definecolor{Gray}{rgb}{0.9,0.9,0.9}
\definecolor{White}{rgb}{1,1,1}
\definecolor{Blue}{rgb}{0.7592156862745099, 0.8454901960784315, 0.9576470588235295}

% Changes title of bibliography
\renewcommand\refname{\textbf{\Large References}}

% Custom commands
\newcommand{\alink}[2]{\href{#1}{\textcolor{blue}{#2}}}

%=========================================================
\begin{document}

\begin{center}
\textbf{\Large Project Description}\\[0.25cm]
\hrulefill\\[0.25cm]
\textbf{\Large Facilitating Community Building in \\[0.2ex] Open Educational Resources \\[0.8ex] for Improved Student Learning Engagement}\\
\hrulefill
\end{center}
\baselineskip 1.5em

%=========================================================
\section{Introduction}

\subsection{What are Open Educational Resources?}
The term \textbf{open educational resources} (OER) was first used in 2002 by a panel of academics convened by UNESCO to discuss primarily the OpenCourseWare (OCW) initiative by MIT \cite{unescoforum:02, oerguidelines}.  MIT's plan was to make freely available online course materials from approximately 2,000 courses for use by anyone worldwide.  There are currently over 2,500 courses on the MIT OpenCourseWare website where for each course one can find a subset of custom notes, textbooks, videos, assignments, exams (some with solutions), and syllabi.  In the last couple of decades, many OER repositories have sprung up such as MERLOT, OER Commons, OASIS, the TU Delft OpenCourseWare, the Open Textbook Library, OpenStax, Lumen Learning, Open Michigan, BCcampus Open Textbook, the Maryland Open Source Textbook project, and many others.  At the center of the OER initiative is the belief that educational resources should be available at no-cost to students and that teaching and learning is strengthened when teaching resources are shared openly by educators.

What are open educational resources?  As defined by UNESCO \cite{oerworldcongress}, OER are teaching, learning, and research materials in any medium that
\begin{compactenum}[(i)]
\item reside in the public domain or
\item have been released under an open license that permits no-cost access, use adaptation, and redistribution by others with no or limited restrictions.
\end{compactenum}
David Wiley from Lumen Learning expands on part (ii) of the definition as the permission to engage in the so-called \textbf{5R} activities \cite{wileynd}:
\begin{compactenum}[(i)]
  \item Retain - make, own, and control a copy of the resource
  \item Revise - edit, adapt, and modify your copy of the resource
  \item Remix - combine your original or revised copy of the resource with other existing material to create something new
  \item Reuse - use your original, revised, or remixed copy of the resource publicly
  \item Redistribute - share copies of your original, revised, or remixed copy of the resource with others
\end{compactenum}

The legal infrastructure supporting OER are copyright licenses that define the level in which users can engage in the 5R activities.  A frequently used open license by OER is one of six possible \textbf{Creative Commons} (CC) licenses \cite{CClicenses:nd}.  A subset of the CC licenses are designed to allow users of creative content to freely distribute, \textbf{adapt}, \textbf{remix}, and \textbf{build upon} the content in any medium or format, possibly for commercial use, provided attribution is given to the original creator.  Open-access or freely available textbooks are sometimes confused with OER but the main difference is that open-access to content may not necessarily grant a user the right to edit, reuse, and redistribute the work.  In other words, the OER movement is not just about making resources available at no-cost but also about making teaching resources available for shared participation and co-creation.  The Open Textbooks Library, one of many OER repositories and hosted by the University of Minnesota, defines an \textbf{open textbook} ``as one that has an open license that makes it free for anyone to use and change. It can be print or digital.''  The stated belief is that ``the ability to make changes to an open textbook is integral to its definition as open'' \cite{opentextbooksfaq:nd}.

\subsection{Cost of College Textbooks: Relief on the Horizon?}
A main reason why educators are initially drawn to the idea of using OER is due to the high costs of college textbooks and the savings that can be passed on to students on adopting OER.  According to data collected and made publicly available by the U.S. Bureau of Labor Statistics (BLS), the price of college textbooks from 1997-2018 increased by an average of nearly 6\% annually which is nearly 3 times the average annual inflation rate of 2\% over the same period \cite{bls}. Only two other consumer services increased higher than college textbooks during the same period and these were hospital services and college tuition \cite{perry2018, perry2022}. %; see Figures~\ref{fig:cpi-textbooks-2018}-\ref{fig:cpi-textbooks-2022}.
In early 2017, the price of college textbooks began to flattened and then declined moderately in 2019 and 2020, remaining flat in 2021 and now experiencing increases throughout 2022.
% \begin{figure}[t]
% \begin{minipage}[t]{8.0cm}
% \includegraphics[width=70mm,height=80mm]{cpichart2018a.png}
% \caption{\small From January 1997 to June 2018, the price of college textbooks has increased by approximately 204\%. (source: \alink{aei.org/carpe-diem}{aei.org/carpe-diem}) }\label{fig:cpi-textbooks-2018}
% \end{minipage}
% \hfill
% \begin{minipage}[t]{8.0cm}
% \includegraphics[width=70mm,height=82mm]{cpi2022junea-3.png}
% \caption{\small From January 2000 to June 2022, the price of college textbooks has increased by approximately 162\%. (source: \alink{aei.org/carpe-diem}{aei.org/carpe-diem})}\label{fig:cpi-textbooks-2022}
% \end{minipage}
% \end{figure}

Despite the moderate decrease in college textbook prices in recent years, and the shift by for profit publishers to subscription-based business models, textbook prices remain a substantial burden for students.  In a series of broadly cited surveys conducted by the Florida Virtual Campus on students from Florida's public colleges and universities (FLVC)\footnote{Surveys were conducted by FLVC in 2010, 2012, 2016, 2018, and 2022; the 2020 survey was cancelled due to the COVID pandemic.}, more than half (53.5\%) of all 13,000 student respondents in 2022 reported that they had not purchased a required textbook due to its cost \cite{flvc2022}. In 2018, of the 22,000 respondents, 66.6\% reported the same.  Other key findings of the 2022 survey found that the high cost of textbooks caused students to:
\begin{compactitem}
\item take fewer courses (43.7\%),
\item not register for a specific course (38.5\%),
\item earn a poor grade due to not being able to afford the textbook (32.4\%), and
\item dropped out of a course (24.2\%).
\end{compactitem}
Students in bachelor degree programs were more likely to spend over \$300 per term on textbooks compared to graduate students or students pursuing an associates degree \cite{flvc2022}.  

\subsection{The Efficacy of OER on Improved Student Outcomes}
While the cost savings for students, educators, and institutions of OER by itself can be a important factor for improved student success and relief for taxpayers \cite{TB-JR-JH:13, RF-RP-BD:15, CW-DD-SC:17}, other benefits of OER include (1) the ability to customize the content to meet the needs of students, (2) the ability to share the content with other educators for improvement, (3) giving students first-day access to course materials which research shows substantially improves learning outcomes \cite{LA:2017}, and (4) giving students the flexibility to engage with the course materials when and where they choose or are able to.  In particular, for students in urban areas that commute to school on public transit, the ability to access course materials on a mobile device is a significant benefit over traditional hard-copy textbooks \cite{CC:17, MS:14}.

Many studies have been performed measuring the efficacy of OER on student performance with mixed results.  Some studies have shown that the adoption of OER result in higher student grades, higher pass rates, lower failing and withdrawal rates, and increased engagement and interested in the subject \cite{AF-MM:12, RF-RP-BD:15, LF-JH:15, RP:15, NP-DB:13, CB-WC-PH:18}.  One large scale study of 21,822 students at the University of Georgia, found that the usage of OER had a significant impact on student performance for students with high financial need (those eligible for a Federal Pell Grant) and for students at greater risk of withdrawing from college such as part-time students \cite{CB-WC-PH:18}.  Of the 5,427 Federal Pell Grant recipients in the study, the number of students in courses adopting OER and the number enrolled in courses using traditional textbooks were approximately event split (2,960 in non-OER and 2,467 in OER courses).  The percentage of students enrolled in courses adopting OER receiving a A, A-, B+, B, or B- in their final grade was 78.4\% while the percentage of students enrolled in courses adopting traditional textbooks receiving a A, A-, B+, B, or B- in their final grade was 65.7\% \cite[pg. 267]{CB-WC-PH:18}.  Similar results were found for part-time students and non-White students (excluding Asian students); final course grades were statistically significantly higher for students enrolled in OER courses compared to students enrolled in traditional textbooks courses \cite{CB-WC-PH:18}.  Moreover, DFW rates for Pell eligible recipients decreased by 4.43\% for students enrolled in courses adopting OER.  And for part-time students, DFW rates decreased by nearly 30\% and average grades increased by 53.12\% for students enrolled in OER courses when compared to students enrolled courses using a traditional expensive notebook.  Overall, the authors of the student note that ``one would not necessarily anticipate that OER would positively impact the performance of a student who would have otherwise been able to purchase a traditional commercial textbook; however, one would imagine that a free textbook would indeed help those students who might choose to forgo a textbook in a course due to the cost'' \cite{CB-WC-PH:18}.

\begin{table}
\centering
\begin{tabular}{crrrr}
  & \multicolumn{2}{c}{Non-Pell Recipients} & \multicolumn{2}{c}{Pell Recipients} \\ \hline
  \multicolumn{1}{c}{Grade} & \multicolumn{1}{c}{Non-OER} & \multicolumn{1}{c}{OER} & \multicolumn{1}{c}{Non-OER} & \multicolumn{1}{c}{OER}\\ \hline
  A\hspace{1.1ex} & 19.48 & 24.90 & 13.48 & 18.97 \\
  A-\hspace{0.7ex} & 11.72 & 19.83 & 10.17 & 16.66 \\
  B+ & 13.70 & 13.90 & 10.88 & 14.84 \\
  B\hspace{1.1ex} & 22.49 & 16.46 & 20.95 & 18.77 \\
  B-\hspace{0.7ex} & 8.92 & 7.54 & 10.20 & 9.16 \\
  C+ & 6.30 & 3.87 & 8.11 & 4.01 \\
  C\hspace{1.1ex} & 6.88 & 5.20 & 10.30 & 6.65 \\
  C-\hspace{0.8ex} & 0.89 & 0.72 & 1.35 & 0.81 \\
  DFW & 9.62 & 7.57 & 14.56 & 10.13 \\ \hline
\end{tabular}
\caption{Student Grade Distribution Based on Pell Eligibility in non-OER and OER Courses \cite{CB-WC-PH:18}}
\end{table}

\begin{table}
\centering
\begin{tabular}{crrrr}
  & \multicolumn{2}{c}{White Students} & \multicolumn{2}{c}{Non-White Students} \\ \hline
  \multicolumn{1}{c}{Grade} & \multicolumn{1}{c}{Non-OER} & \multicolumn{1}{c}{OER} & \multicolumn{1}{c}{Non-OER} & \multicolumn{1}{c}{OER}\\ \hline
  A\hspace{1.1ex} & 20.22 & 26.27 & 11.83 & 15.96 \\
  A-\hspace{0.7ex} & 12.51 & 19.95 & 8.33 & 17.23 \\
  B+ & 13.85 & 14.65 & 10.45 & 13.91 \\
  B\hspace{1.1ex} & 22.42 & 16.05 & 22.08 & 19.52 \\
  B-\hspace{0.7ex} & 8.91 & 7.54 & 10.40 & 8.44 \\
  C+ & 5.96 & 3.24 & 9.27 & 5.47 \\
  C\hspace{1.1ex} & 6.59 & 4.48 & 10.89 & 8.10 \\
  C-\hspace{0.8ex} & 0.85 & 0.62 & 1.48 & 1.22 \\
  DFW & 8.70 & 7.19 & 15.28 & 10.15 \\ \hline
\end{tabular}
\caption{Student Grade Distribution Based on Ethnicity in non-OER and OER Courses \cite{CB-WC-PH:18}}
\end{table}

\begin{table}
\centering
\begin{tabular}{crrrr}
  & \multicolumn{2}{c}{Full-Time Students} & \multicolumn{2}{c}{Part-Time Students} \\ \hline
  \multicolumn{1}{c}{Grade} & \multicolumn{1}{c}{Non-OER} & \multicolumn{1}{c}{OER} & \multicolumn{1}{c}{Non-OER} & \multicolumn{1}{c}{OER}\\ \hline
  A\hspace{1.1ex} & 20.25 & 23.70 & 6.28 & 18.70 \\
  A-\hspace{0.7ex} & 12.67 & 19.47 & 4.45 & 10.98 \\
  B+ & 14.05 & 14.41 & 7.54 & 8.74 \\
  B\hspace{1.1ex} & 22.85 & 17.15 & 18.26 & 14.43 \\
  B-\hspace{0.7ex} & 9.11 & 7.80 & 9.94 & 10.57 \\
  C+ & 6.32 & 3.87 & 9.00 & 4.67 \\
  C\hspace{1.1ex} & 7.48 & 5.49 & 9.11 & 6.71 \\
  C-\hspace{0.8ex} & 0.99 & 0.73 & 1.10 & 1.02 \\
  DFW & 6.28 & 7.38 & 34.33 & 24.19 \\ \hline
\end{tabular}
\caption{Student Grade Distribution Based on Registration Status in non-OER and OER Courses \cite{CB-WC-PH:18}}
\end{table}

\subsection{Barriers in OER Adoption}
Recent studies show that awareness of OER in colleges and universities has grown in the last decade driven primarily by statewide and/or institutional initiatives.  For example, in annual national surveys conducted by Bay View Analytics, which collects a representative sample of the broad range of teaching faculty in U.S. higher education for their surveys, 46\% of polled faculty indicated being aware of both OER and CC licensing in the 2021-2022 survey compared to only 17\% in 2014-2015 \cite{JS-JS:2022}; see Figure~\ref{fig:oer-awareness}.  In the same survey \cite{JS-JS:2022}, the majority of respondents indicated that, because of the switch to online learning due to the COVID pandemic, their opinions of online learning improved (54\%) as did their acceptance of digital materials (68\%).  Although progress has been made in increasing college faculty awareness of OER, much work remains to be done to increase the actual usage and creation of quality open educational resources \cite{flvc2022}.
\begin{figure}[t]
\centering
\includegraphics[width=90mm]{oer_awareness.png}
\caption{The percent of U.S. higher education teaching faculty who indicated being aware of both open educational resources (OER) and Creative Commons licenses by year \cite{JS-JS:2022}.}
\label{fig:oer-awareness}
\end{figure}

Despite the growing number of educators and institutions embracing the usage of OER in higher-education teaching and learning, the actual usage of, and participation in, OER is still somewhat limited \cite{MB:2022}.  In particular, even in communities where OER is adopted, the rights of users to engage in the 5R activities (as described in CC licenses) are often hampered by technological barriers \cite{CC:16, SO:19, MM-EC:19, CE:22}:
\begin{quote}\em
  The truth is that while most open textbooks are legally licensed to be modified, the real-life work involved can sometimes be tough because of technical issues. \cite{CC:16}
\end{quote}
A recent study of open courseware (OCW) on two large OCW platforms (MIT and Delft University of Technology) found that a random sample of ten OCW courses were less open than expected in terms of several factors.  In the study, a previously developed framework was used to evaluate the openness of OER using eight decision factors \cite{MM-EC:19}; one critical factor being the \textbf{file format} used to distribute the content \cite{CE:22}. Of the 10 courses analyzed in the study, all courses were categorized as \textbf{closed} when using the file format as the decision factor to measure openness (see Table~\ref{tab:open-enough}).
\begin{table}
\centering
\renewcommand{\arraystretch}{1.5}
\begin{tabular}{p{5cm}p{5cm}p{5cm}}\hline  
  \rowcolor{Gray}
  \multicolumn{1}{c}{\textbf{Closed}} & \multicolumn{1}{c}{\textbf{Mixed}} & \multicolumn{1}{c}{\textbf{Most Open}}\\ \hline 
A print resource, document image, PDF, or other non-editable format that cannot be altered without expensive software & Editable proprietary file format that could be adapted using open software (e.g. .docx file edited using LibreOffice) & Fully open format (e.g. HTML or .ODF) that could be edited using either open or proprietary software\\[1ex] \hline
\end{tabular}
\caption{The three levels of openness on file format defined as part of the Open Enough framework measuring openness in OER and OCW \cite{MM-EC:19}.}
\label{tab:open-enough}
\end{table}
A principal conclusion made in the study was that OCW were open from a licensing perspective but not well suited for educator adaption which is one of the underlying principles of the OER movement.  In the courses analyzed, the text content of the course was provided almost exclusively in non-editable PDF format.  A similar analysis was performed on the STEM textbooks catalogued by the Open Textbook Library and the results were similar (see Figure~\ref{fig:open-book-formats}); a large portion of the textbooks were available in non-editable format only \cite{CA:22}.  For textbooks containing large amounts of mathematical content, conversion from PDF to other more editable formats is difficult at best or impossible.  From an educators perspective, the use of PDF alone, or other non-editable file format, severely limits or completely eliminates the ability to adapt the material to a local learning environment.
\begin{figure}
  \centering
  \includegraphics[width=80mm]{stem-textbook-formats.png}
  \caption{On the repository Open Textbook Library, the number of unique textbooks in STEM subjects (Natural Sciences, Computer Science, Engineering, and Mathematics) is 338.  Each textbook is available in at least one format (PDF, HTML, Hardcopy, LaTeX, eBook, MS Word, Google Doc, XML, ODF). Essentially all books are available in PDF (99\%) and nearly half (48\%) are available in HTML.  Data compiled on 12-16-2022 from \alink{open.umn.edu/opentextbooks}{open.umn.edu/opentextbooks}.}
  \label{fig:open-book-formats}
\end{figure}

The file format chosen to distribute OER affects not only the adaptability of course materials but also has implications on the usability of education materials for learners with visual, auditory, or physical disabilities, and in certain situational environments.  According to the W3C Web Accessibility Initiative which develops accessibility standards and guidelines for the web \cite{w3cwai}, ``web accessibility means that websites, tools, and technologies are designed and developed so that people with disabilities can use them'' and that people with disabilities should be able to ``perceive, understand, \textbf{navigate}, and \textbf{interact} with the Web and \textbf{contribute} to the Web''.  However, web accessibility is also important for people without disabilities, for example, for people using the web on small screen sizes such as mobile phones or tablets, for people with slow internet connections or limited bandwidth, or for people in situational environments where lighting and sound are limiting factors to web content consumption.  With regards to educational materials, file formats such as PDF impose significant limitations to interacting with and navigating web content.  According to research performed by the Norman Nielsen Group  web user-experience firm \cite{JN-AK:20}, forcing users to browse PDF files makes usability approximately 300\% worse compared to HTML pages.  Problems with using PDF to deliver and consume content on the web are:
\begin{compactitem}
\item PDF files are optimized for paper sizes, not browser windows or modern screen sizes, 
\item PDF files can be quite large thereby increasing download times when in many cases a user may only need immediate read-access to only part of the content and not the entire document at once,
\item PDF files are not inherently accessible and can be difficult to navigate especially with small screen sizes or users with disabilities,
\item PDF files are closed for editing and adaptation without expensive software, and
\item without additional effort by content creators, screen readers cannot reliably parse special elements in a PDF file let alone elements that contain rich mathematical content.
\end{compactitem}
The limitations with using PDF to deliver web content not only affects user experience but also users' perceived \textbf{quality} of the content which studies have shown is one of the barriers to OER adoption by faculty \cite{OB-RB:16, RJ-RP-CH:2016, JS-JS:2018}.

In contrast, HTML is \textit{the} native mark-up language of the web and is the language supported by all web browsers.  HTML is inherently accessible and can be made more accessible with the use of HTML5 semantic tags and special attributes.  Moreover, recent technologies such as MathJax have made it possible to embed mathematical mark-up languages such as LaTeX directly inside HTML documents to create web pages that contain rich mathematical content.  The ability to embed LaTeX in HTML has profound implications on the creation of web-based OER since LaTeX is a popular mark-up language used by STEM authors in mathematics, engineering, computer science, and physics.  However, since many authors use LaTeX to write course notes, textbooks, and other educational materials, content creators are faced with the difficult and time-consuming task of converting LaTeX source code to HTML in order to take advantage of the native rendering capabilities in web browsers.  Although existing open-source software exist to convert between different mark-up languages such as LaTeX and HTML, most notably \textsf{pandoc}, the conversion process is not always straightforward and requires a certain level of technical expertise.  Alternatively, PreTeXt is a recent mark-up language for authoring and publishing content on the web (and other formats) especially in STEM disciplines but requires one to learn a new mark-up language and also requires the conversion of content already written in LaTeX to PreTeXt.

%=========================================================
\section{Project Proposal}

A promise of open educational resources (OER) is that by removing barriers to education such as cost, accessibility, quality, and relevancy to the local learning environment, OER can help improve student outcomes and learning experiences.  A principal belief of OER is that student success can be improved when users of OER can actively participate in the creation and adaptation of educational materials for one's own consumption and for future downstream learners.  Although OER initiatives at global, national, state, and institutional levels have increased awareness of the benefits of OER, the actual practical integration and usage of OER in higher-education is still somewhat limited.  Nonetheless, more and more educators, initially drawn by the cost savings passed on to students by adopting OER, are rethinking their approaches to teaching thanks to the ecosystem surrounding the OER movement and are fostering more student-centered, learner-sourced, community-based, and diversity-infused learning environments.

The goal of this project is to advance undergraduate STEM education by building communities of practice around the creation, usage, improvement, and distribution of open educational resources in undergraduate mathematics courses.  Students at a liberal arts college in varying levels of study and disciplines will engage in learner-sourcing group activities to increase student engagement in the learning process with the aim of solidifying the understanding of learnt material.  Studies have shown that student learning outcomes improve when students are active participants in the creation of learning materials rather than being solely consumers of information.  Students will generate assessments for their peers and simultaneously answer and assess assessments created by their peers. The creation of the assessments will be guided by students' own misconceptions of the content and through self-reflective questioning of perceived understanding and thus also improving the learning experience of future students.  In the process, under-represented students in STEM and future high-school teachers will be trained to use mathematical mark-up languages and version control systems and thereby increasing the number of computer-literate individuals entering the education sector.  A main part of the project will be the development and deployment of a web-based software tool to facilitate the creation of STEM open educational resources with a primary goal of facilitating the reusability of educational materials and increasing the availability of no-cost high-quality resources for instructors and students.  Quantitative and qualitative data will be collected from students to evaluate learning outcomes and perceptions of learner-sourcing activities.  An overarching goal of the project is to implement a model of closer collaboration between educators and students in the creation, delivery, and consumption of STEM education.  Results of the project will be disseminated in national and regional workshops and software tools will be made freely available to the general public.


\subsection{Overview}

%=========================================================
\subsection{Existing Technology}

\subsubsection{MathML}
\textbf{Mathematical Markup Language} (MathML) is a language used to describe the presentation and content of mathematical notation.  MathML is used in web browsers, in computer algebra systems (CAS), print typesetting, and voice synthesis.  The goal of MathML is to enable mathematically rich documents on the web in the same way that HTML has enabled this functionality for text; MathML markup can be written alongside HTML markup.  MathML markup can become quite complicated even for simple mathematical expressions.  As an example, the MathML markup to describe the mathematical expression
\[
x = \frac{-b \pm\sqrt{b^2-4ac}}{2a}
\]
is given by
\begin{verbatim}
    <math>
      <mi>x</mi><mo>=</mo>
      <mfrac>
        <mrow>
          <mo>-</mo><mi>b</mi><mo>&pm;</mo>
          <msqrt>
            <msup><mi>b</mi><mn>2</mn></msup>
            <mo>-</mo><mn>4</mn><mi>a</mi><mi>c</mi>
          </msqrt>
        </mrow>
        <mrow><mn>2</mn><mi>a</mi></mrow>
      </mfrac>
    </math>
\end{verbatim}
whereas the equivalent LaTeX markup is
\begin{verbatim}
    \[
        x = \frac{-b \pm \sqrt{b^2-4ac}}{2a}
    \]
\end{verbatim}
Needless to say, typing MathML by hand can be very tedious and error prone and as such it is not intended to be edited by hand.  Instead, the generation of MathML markup is usually the task of specialized equation editors or the result of a conversion from another markup language such as LaTeX.
 
\subsubsection{LaTeX2HTML}
\textbf{LaTeX2HTML} is a command line utility that converts LaTeX documents to web pages in HTML (latex2html.org).  As described on the programs website, ``\textit{LaTeX2HTML replicates the basic structure of a LaTeX document as a set of interconnected HTML files which can be explored using automatically generated navigation panels. The cross-references, citations, footnotes, the table of contents and the lists of figures and tables, are also translated into hypertext links. Formatting information which has equivalent tags in HTML (lists, quotes, paragraph breaks, type styles, etc.) is also converted appropriately. The remaining heavily formatted items such as mathematical equations, pictures or tables are converted to images which are placed automatically at the correct positions in the final HTML document''.}\footnote{latex2html.org}  Although LaTeX2HTML's goal of converting LaTeX documents to web pages is similar in spirit to htexml, there are core design features that distinguish LaTeX2HTML and htexml, such as:

\begin{itemize}
\item Documentation for LaTeX2HTML is scarce and not easily accessible for an average computer user.  Basic usage of LaTeX2HTML is not available on the projects website or github page.
\item LaTeX2HTML's conversion of mathematical equations to images to be rendered by the browser is in conflict with the goal of creating mathematically rich documents on the web that are rendered natively by the browser.
\item LaTeX2HTML is written in \textbf{perl} and must be installed from source using using configure, make, and make install or is available as a package for Linux or through the MacOS package manager Homebrew.
\item After installation, the program is available as a command line utility whereas all the functionality of htexml is accessible from a browser.
\item Even for an experienced computer user, installing LaTeX2HTML from source or via Homebrew can present its challenges that could discourage users from using the program.  htexml requires no installation.
\end{itemize}

\subsubsection{TeX4ht}
\textbf{TeX4ht} is a command line tool that converts LaTeX documents to various output formats including HTML, ODT (open document format), or DocBook.  According to the programs documentation, modern TeX distributions such as TeX Live and Miktex come with TeX4ht and a related bundling tool called \textbf{make4ht}.  



%=========================================================
\subsection{Placeholder}


%=========================================================
\section{Research Plan and Timeline}


\noindent\textbf{First Year:}  

\noindent\textbf{Second Year:}  

\noindent\textbf{Third Year:}  
%=============================================
\section{Broader Impacts}

\subsection{Research Impact}


%=============================================
\subsection{Training of STEM Students}

\subsection{Educational Impact}    


%=============================================
\newpage

% Bibliography
\begin{thebibliography}{99}

  \bibitem{CA:22} Aguilar, C.O. (2022). Using file format to measure openness in open educational resources. In preparation.

  \bibitem{LA:2017} Agnihotri, L., Essa, A., \& Baker, R. (2017, March). Impact of student choice of content adoption delay on course outcomes. {\em In Proceedings of the Seventh International Learning Analytics \& Knowledge Conference} (pp. 16-20)

  \bibitem{MB:2022} Baas, M., van der Rijst, R., Huizinga, T., van den Berg, E., Admiraal, W. (2022). Would you use them? A qualitative study on teachers' assessments of open educational resources in higher education. {\em The Internet and Higher Education}, 54, 100857.

  \bibitem{OB-RB:16} Belikov, O. M., and Bodily, R. (2016). Incentives and barriers to OER adoption: A qualitative analysis of faculty perceptions. Open Praxis, 8(3), 235–246. %DOI: \alink{http://doi.org/10.5944/openpraxis.8.3.308}{http://doi.org/10.5944/openpraxis.8.3.308}
  
  \bibitem{TB-JR-JH:13} Bliss, T., Robinson, T.J., Hilton, J. and Wiley, D.A., 2013. An OER COUP: College Teacher and Student Perceptions of Open Educational Resources. Journal of Interactive Media in Education, 2013(1), p.Art. 4.% DOI: \alink{http://doi.org/10.5334/2013-04}{http://doi.org/10.5334/2013-04}


  \bibitem{CC:17} Cooney, C. (2017). What impacts do OER have on students? Students share their experiences with a health psychology OER at New York City College of Technology. {\em International Review of Research in Open and Distributed Learning}, 18(4), 155-178.

  \bibitem{CE:22} Christiansen, E. and McNally, M. (2022). Examining the technological and pedagogical elements of select open courseware. {\em First Monday}. 27, 10.% \alink{https://doi.org/10.5210/fm.v27i10.11639}{https://doi.org/10.5210/fm.v27i10.11639}

  \bibitem{CClicenses:nd} About CC Licenses (n.d.), Creative Commons. (n.d.). Retrieved from \alink{https://creativecommons.org/about/cclicenses/}{https://creativecommons.org/about/cclicenses/}

  \bibitem{VC-SK:19} Clinton, V., and Khan, S. (2019). Efficacy of Open Textbook Adoption on Learning Performance and Course Withdrawal Rates: A Meta-Analysis. {\em AERA Open}, 5(3).% DOI: \alink{https://doi.org/10.1177/2332858419872212}{https://doi.org/10.1177/2332858419872212}

\bibitem{CB-WC-PH:18} Colvard, N.\/B., Watson C.\/ E., and Park, H. (2018). The impact of open educational resources on various student success metrics. {\em International Journal of Teaching and Learning in Higher Education, 30(2), 262–276}. Retrieved from

  \bibitem{CC:16} Cuillier, C., Hofer, A., Johnson, A., Labadorf, K., Lauritsen, K., Potter, P., \& Walz, A. (2016). Modifying an open textbook: What you need to know.  {\em Pressbooks}.%  \alink{https://pressbooks.pub/oenmodify/}{https://pressbooks.pub/oenmodify/}

  \bibitem{RF-RP-BD:15} Farrow, R., Pitt, R., de los Arcos, B., Perryman, L.A., Weller, M. and McAndrew, P. (2015). Impact of OER use on teaching and learning: data from OER Research Hub (2013–2014). {\em British Journal of Educational Technology}. 46(5) pp. 972–976.

  \bibitem{unescoforum:02} Forum on the Impact of Open Courseware for Higher Education in Developing Countries, UNESCO, Paris, 1-3 July 2002: final report.  UNESDOC Digital Library.%  \alink{https://unesdoc.unesco.org/ark:/48223/pf0000128515.locale=en}{https://unesdoc.unesco.org/ark:/48223/pf0000128515.locale=en}

  \bibitem{opentextbooksfaq:nd} Frequently Asked Questions (n.d.). Open Textbook Library. Retrieved from \alink{https://open.umn.edu/opentextbooks/faq}{https://open.umn.edu/opentextbooks/faq}

  \bibitem{AF-MM:12} Feldstein, A., Martin, M., Hudson, A., Warren, K., Hilton III, J., and Wiley, D. (2012). Open textbook and increased student access and outcomes. {\em European Journal of Open, Distance, and ELearning}, 15(2).

  \bibitem{LF-JH:15} Fischer, L., Hilton III, J., Robinson, T. J., and Wiley, D.
  A. (2015). A multi-institutional study of the impact of open textbook adoption on the learning outcomes of post-secondary students. {\em Journal of Computing in Higher Education}, 27(3), 159–172.

  \bibitem{hilton:16} Hilton J. III. (2016). Open educational resources and college textbook choices: A review of research on efficacy and perceptions. {\em Educational Technology Research and Development}, 64, 573–590.

  \bibitem{hilton:20} Hilton, J. (2022). Open educational resources, student efficacy, and user perceptions: a synthesis of research published between 2015 and 2018. {\em Education Tech Research Dev}, 68, 853–876 (2020).% \alink{https://doi.org/10.1007/s11423-019-09700-4}{https://doi.org/10.1007/s11423-019-09700-4}.

  \bibitem{RJ-RP-CH:2016} Jhangiani R., Pitt R., Hendricks C., Key J., Lalonde C. (2016). Exploring faculty use of open educational resources in B.C. post-secondary institutions. {\em BCcampus}. %Retrieved from \alink{https://bccampus.ca/2016/01/27/exploring-faculty-use-of-open-educational-resources-in-b-c-post-secondary-institutions/}{https://bccampus.ca/2016/01/27/exploring-faculty-use-of-open-educational-resources-in-b-c-post-secondary-institutions/}

  \bibitem{w3cwai} Introduction to Web Accessibility.  W3C Web Accessibility Initiative. \alink{w3.org}{w3.org}%  Retrieved from \alink{https://www.w3.org/WAI/fundamentals/accessibility-intro/}{https://www.w3.org/WAI/fundamentals/accessibility-intro/}% Accessed on 12-20-2022.

  \bibitem{MM-EC:19} McNally, M. B., and Christiansen, E. G. (2019). Open enough? Eight factors to consider when transitioning from closed to open resources and courses: A conceptual framework. {\em First Monday}, 24(6).% \alink{https://doi.org/10.5210/fm.v24i6.9180}{https://doi.org/10.5210/fm.v24i6.9180}

  \bibitem{oerguidelines} Miao, F., Mishra, S., Orr, D., \& Janssen, B. (2019). Guidelines on the development of open educational resources policies. {\em UNESCO Publishing}.%\newline \alink{https://unesdoc.unesco.org/ark:/48223/pf0000371129}{https://unesdoc.unesco.org/ark:/48223/pf0000371129}

  \bibitem{JN-AK:20} Nielsen, J. and Kaley, A. (2020).  Avoid PDF for On-Screen Reading, {\em Nielsen Norman Group}.% \alink{https://www.nngroup.com/articles/avoid-pdf-for-on-screen-reading/}{https://www.nngroup.com/articles/avoid-pdf-for-on-screen-reading/} Accessed 12-20-2022.

  \bibitem{oerworldcongress} Second World OER Congress: Ljubljana OER Action Plan. (2017). UNESCO. \newline Retrieved from \alink{https://unesdoc.unesco.org/ark:/48223/pf0000260762}{https://unesdoc.unesco.org/ark:/48223/pf0000260762}

  \bibitem{wileynd} Wiley, D. (n.d.). Defining the ``open'' in open content and open educational resources.\newline Retrieved from \alink{http://opencontent.org/definition/}{http://opencontent.org/definition/}.

  \bibitem{JS-JS:2022} Seaman, J.E., Seaman, J. (2022). Turning Point for Digital Curricula: EducationAL Resources in U.S. Higher Education. {\em Bay View Analytics}.% \newline \alink{https://www.bayviewanalytics.com/reports/turningpointdigitalcurricula.pdf}{https://www.bayviewanalytics.com/reports/turningpointdigitalcurricula.pdf}

  \bibitem{JS-JS:2018} Seaman, J.E., Seaman, J. (2002). Freeing the Textbook: Educational Resources in U.S. Higher Education. {\em Bay View Analytics}. %\newline \alink{https://www.bayviewanalytics.com/reports/freeingthetextbook2018.pdf}{https://www.bayviewanalytics.com/reports/freeingthetextbook2018.pdf}

  \bibitem{SO:19} Ovadia, S. (2019). Addressing the Technical Challenges of Open Educational Resources.  {\em portal: Libraries and the Academy}, 19(1), 79-93.% \alink{http://doi.org/10.1353/pla.2019.0005}{http://doi.org/10.1353/pla.2019.0005}

  \bibitem{bls} U.S. Bureau of Labor Statistics, {\em Consumer Price Index for All Urban Consumers: College textbooks in U.S. city average}, Series ID CUUR0000SSEA011.% \alink{https://data.bls.gov/cgi-bin/srgate}{https://data.bls.gov/cgi-bin/srgate}

  \bibitem{NP-DB:13} Pawlyshyn, N., Braddlee, D., Casper, L., and Miller, H. (2013). Adopting OER: A case study of crossinstitutional collaboration and innovation. {\em EDUCAUSE Review}.

  \bibitem{perry2018} Perry, M.J. (2018). The CD `chart of the Century' Makes the Rounds at the Federal Reserve. {\em Carpe Diem}.% \alink{https://www.aei.org/carpe-diem/the-chart-of-the-century-makes-the-rounds-at-the-federal-reserve/}{https://www.aei.org/carpe-diem/the-chart-of-the-century-makes-the-rounds-at-the-federal-reserve/}.

  \bibitem{perry2022} Perry, M.J. (2022). Chart of the Day \ldots or Century?. {\em Carpe Diem}.% \alink{https://www.aei.org/carpe-diem/chart-of-the-day-or-century-8/}{https://www.aei.org/carpe-diem/chart-of-the-day-or-century-8/}.

  \bibitem{RP:15} Pitt, R. (2015). Mainstreaming Open Textbooks: Educator Perspectives on the Impact of OpenStax College open textbooks. {\em The International Review of Research in Open and Distributed Learning}, 16(4).

  \bibitem{MS:14} Smale, M.A., \& Regalado, M. (2014). Commuter students using technology. {\em EDUCAUSE Review Online}. Retrieved from \alink{http://www.educause.edu/ero/article/commuter-students-using-technology}{http://www.educause.edu/ero/article/commuter-students-using-technology}

  \bibitem{flvc2022} Florida Virtual Campus, 2022 Student Textbook and Instructional Materials Survey. Tallahassee, FL. %\alink{https://dlss.flvc.org/colleges-and-universities/research/textbooks}{https://dlss.flvc.org/colleges-and-universities/research/textbooks}.

  \bibitem{CW-DD-SC:17} Watson, C. E., Domizi, D. P., and Clouser, S. A. (2017). Student and Faculty Perceptions of OpenStax in High Enrollment Courses. {\em The International Review of Research in Open and Distributed Learning}, 18(5). %\alink{https://doi.org/10.19173/irrodl.v18i5.2462}{https://doi.org/10.19173/irrodl.v18i5.2462}

\end{thebibliography}


\end{document}
